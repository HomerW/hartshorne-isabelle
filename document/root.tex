\documentclass[11pt,notitlepage,openany]{book}
\usepackage{isabelle,isabellesym,eufrak}
\usepackage[english]{babel}
\usepackage[svgnames]{xcolor}
\usepackage{amsmath, amsthm, amssymb}
\usepackage{listingsutf8, amstext, wasysym}
\usepackage{isabelle-listings}

% For graphics files
%\usepackage[pdftex]{graphicx}
\usepackage[framemethod=default]{mdframed}
\mdfdefinestyle{exampledefault}{%
linecolor=blue, linewidth=2pt,%
rightline=true, leftmargin=0.3cm,
topline=false, bottomline=false, rightline=false}

\newtheorem{thm}{Theorem}[section]
\newtheorem{corr}[thm]{Corollary}
\newtheorem{lemma}[thm]{Lemma}
\newtheorem{prop}[thm]{Lemma}

\theoremstyle{definition}
\newtheorem{defn}[thm]{Definition}

% \begin{thm}\label{...} .... \end{thm}
% \begin{corr}\label{...}  .... \end{cor}
% \begin{lemma}\label{...}  .... \end{lem}
% \begin{prop}\label{...}  .... \end{lem}

\newenvironment{hartshorne}%
  {\begin{mdframed}[style=exampledefault]}%
  {\end{mdframed}}%
%



\newcommand{\spike}  {{~\newline \color{red}\rule{1cm}{0.2cm} }}
\newcommand{\david}  {{\\ \color{brown}\rule{1cm}{0.2cm} }}
\newcommand{\ken}    {{\\ \color{green}\rule{1cm}{0.2cm} }}
\newcommand{\jackson}{{\\ \color{lime}\rule{1cm}{0.2cm} }}
\newcommand{\daniel} {{\\ \color{magenta}\rule{1cm}{0.2cm} }}
\newcommand{\james}  {{\\ \color{violet}\rule{1cm}{0.2cm} }}
\newcommand{\justin} {{\\ \color{red}\rule{1cm}{0.2cm} }}
\newcommand{\petar}  {{\\ \color{LightSalmon}\rule{1cm}{0.2cm} }}
\newcommand{\seiji}  {{\\ \color{DeepSkyBlue}\rule{1cm}{0.2cm} }}
\newcommand{\caleb}  {{\\ \color{Tan}\rule{1cm}{0.2cm} }}
\newcommand{\homer}  {{\\ \color{orange}\rule{1cm}{0.2cm} }}
\newcommand{\siqi}   {{\\ \color{cyan}\rule{1cm}{0.2cm} }}
\newcommand{\haoze}  {{\\ \color{blue}\rule{1cm}{0.2cm} }}
\newcommand{\done}   {{~\newline \color{black}\rule{1cm}{0.2cm} }}



% this should be the last package used
\usepackage{pdfsetup}

% urls in roman style, theory text in math-similar italics
\urlstyle{rm}
\isabellestyle{it}

\begin{document}

\title{Foundations of Projective Geometry: A formalization}
\author{John Hughes and the students of Brown CS1951D, Spring 2020}
\maketitle

%\begin{abstract}
%Starting from Robin Hartshorne's book, we prove things.
%\end{abstract}

% \tableofcontents

\chapter*{Introduction}
This text is a formalization of Robin Hartshorne's \emph{Foundations of Projective Geometry}
using the Isabelle proof assistant, primarily relying on Isar. Quotations 
from Hartshorne appear indented, with a blue line to the left. Additional material 
written by either the professor (John (Spike) Hughes) or various students are surrounded by colored 
markers, like this:
\spike
This is something written by Spike
\done
with the black marker indicating the end of the section written by Spike (except that in this case, 
it's part of a larger section Spike wrote). 

Within Isabelle, numbered propositions or theorems from Hartshorne are given names that tie back 
to the text, so Proposition 1.1 in the text is called \texttt{Prop1P1}, with ``P'' replacing the period, 
for instance. 

Students should insert things into the document using
macros associated with their name (in lowercase) to produce a marker indicating the start of their contribution, and the "done" macro to indicate when their contribution is complete. Here are examples of the macro results:

\spike Spike
\david David
\ken Ken
\jackson Jackson
\daniel Daniel
\brad Brad
\justin Justin
\petar Petar
\seiji Seiji
\caleb Caleb
\homer Homer
\siqi Siqi
\haoze Haoze
\done the ``done'' macro.


\input{session}

\bibliographystyle{abbrv}
\bibliography{root}

\end{document}
